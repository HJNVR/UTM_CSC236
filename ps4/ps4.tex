% Template to use to complete Problem Set 4.
% If you are using ShareLaTeX, you'll want to upload this file to your account.
% Before modifying this file, we recommend trying to compile it as-is
% to see what the basic template gives.

\documentclass[12pt]{article}
\usepackage{geometry}
\geometry{letterpaper}
\usepackage{amssymb}
\usepackage{amsmath}
\usepackage{graphicx}
\newcounter{ProblemNum}
\newcounter{SubProblemNum}[ProblemNum]
\renewcommand{\theProblemNum}{\arabic{ProblemNum}}
\renewcommand{\theSubProblemNum}{\alph{SubProblemNum}}
\newcommand*{\anyproblem}[1]{\newpage\subsection*{#1}}
\newcommand*{\problem}[1]{\stepcounter{ProblemNum} %
\anyproblem{Problem \theProblemNum. \; #1}}
\newcommand*{\soln}[1]{\subsubsection*{#1}}
\newcommand*{\solution}{\soln{Solution}}
\renewcommand*{\part}{\stepcounter{SubProblemNum} %
\soln{Part (\theSubProblemNum)}}


% Document metadata
\title{Problem Set \#4  \hspace{3cm} CSC236 Fall 2018}
\author{Si Tong Liu, shuo Yang, Jing Huang}
\date{22/11/2018}


% Document starts here
\begin{document}
\maketitle



\noindent \rule{\textwidth}{1pt}





\vfill
We declare that this assignment is solely our own work, and is in accordance
with the University of Toronto Code of Behaviour on Academic Matters.

\noindent \rule{\textwidth}{1pt}

This submission has been prepared using \LaTeX.

\newpage

\problem{}
%%%%%%%%%%%%%%%
%%%%%%%%%%%%%%
\textsc{(Warmup - this problem will NOT be marked)}.
\\

Here is code for a recursive function that finds the minimum element of a list.

\begin{small}
\begin{verbatim}
def rec_min(A):
  if len(A) == 1:
    return A[0]
  else:
    m = len(A) // 2
    min1 = rec_min(A[0..m-1])
    min2 = rec_min(A[m..len(A)-1])
    return min(min1, min2)
\end{verbatim}
\end{small}

State preconditions and postconditions for this function. Then, prove that this algorithm is correct according to your specifications.


%%Write your solution here
Solution: \vskip5pt
1. Precondition: A is a list and len(A) $>$ 0\vskip5pt
2. Postcondition: return the minimum element of list A
3. Prove the correctness of the algorithm: \vskip5pt
    # path 1: return A[0] \vskip5pt
    To enter this path, len(A) == 1, which means there is only one element in listA\vskip5pt
    so the minimum element is the only element in list A, holds\\[2ex]
    
    # path 2: return min(min1, min2)
    To enter this path, len(A) $\ge 2 $ \vskip5pt
    (1) then len(A) // 2 $> 1$ , precondition holds \vskip5pt
    (2) m = len(a) // 2, m is getting smaller, so smaller value holds \vskip5pt
    (3) for min1, assume list1 = A[0..m-1], so len(list1) = m-1-(-1) = m \vskip5pt
        for min2, assume list2 = A[m..len(A) - 1]m so len(list2) = len(A)-1 - (m-1) = len(A)-m \vskip5pt
        so that len(list1) + list(list2) = len(A)-m+m = len(A)\vskip5pt
        which means the minimum of min1 and min2 is indeed the minimum element of listA\vskip5pt
        Then we use complete induction to prove the recursive call \vskip5pt
        We assume the the recursive call holds for listA with length of n
        prove for n+1\vskip5pt
        [1] n is odd
        list1 becomes A[0..m],so len(list1) = m-(-1) = m + 1\vskip5pt
        list2 becomes A[m+1,n+1-1],so len(list2) = n - m\vskip5pt
        so that len(list1) + len(list2) = m + 1 + n -m = n + 1, which is the length of listA
        which means the minimum of min1 and min2 is the minimum element of listA, holds\vskip5pt
        [2] n is even
        list1 remains the same A[0..m], so len(list1) = m-1-(-1) = m\vskip5pt
        list2 becomes A[m..n+1-1], so len(list2) = n-(m-1) = n-m+1\vskip5pt
        so that len(list1) + len(list2) = n-m+1+m = n+1, which is the length of listA\vskip5pt 
        which means the minimum of min1 and min2 is the minimum element of listA, holds\vskip5pt
        
Therefore, the algorithm is correct
        
        
        
        

%%%%%%%%%%%%%%%
%%%%%%%%%%%%%%
\problem{}
\textsc{(10 Marks)} \textbf{Iterative Program Correctness.} One of your tasks in this assignment is to write a proof that a program works correctly, that is,
you need to prove that for all possible inputs, assuming the precondition is satisfied, the postcondition is satisfied after a finite number of steps.


This exercise aims to prove the correctness of the following program:

\begin{small}
\begin{verbatim}
def mult(m,n) : # Pre-condition: m,n are natural numbers
""" Multiply natural numbers m and n """
1.     x = m
2.     y = n
3.     z = 0
4.     # Main  loop 
5.     while not x == 0 :
6.         if x % 3 == 1 : 
7.             z = z + y 
8.         elif x % 3 == 2 : 
9.             z = z - y
10.            x = x + 1
11.        x = x div 3 
12.        y = y * 3
13.    # post condition: z = mn
14.    return z
\end{verbatim}
\end{small}

Let \verb|k| denote the iteration number (starting at 0) of the \verb|Main Loop| 
starting at line 5 ending at line 12. We will denote $I_k$ the
iteration $k$ of the \verb|Main Loop|.  Also for each iteration,  let $x_k,
y_k, z_k$ denote the values of the variables $x,y,z$ at line 5 (the staring
line) of $I_k$.


\begin{enumerate}

\item {(5 Marks)} \textbf{Termination}. Need to prove that for all natural numbers $n, m$, there exist an iteration $k$, such that $x_k=0$ at the beginning of $I_k$, that is at line 5.


\textsc{Hint:} You may find helpful to prove this helper statement first:


For all natural numbers $k$, $x_k > x_{k+1}\geq 0$. (Hint: do not use induction).


\item {(2 Marks)} \textbf{Loop invariant}


Let $P(k)$ be the predicate: At the end of $I_k$ (line 12), $z_k = mn-x_ky_k$.

Using induction, prove the following statement:

$$\forall k\in \mathbb{N}, P(k)$$

\item {(3 Marks)} \textbf{Correctness} Let $C(m, n)$ be the following predicate defined in the domain of natural numbers: Program \verb|mult(m,n)| returns $mn$. Let $S(m, n)$ be the function equal to the number of steps of the program \verb|mult(m, n)|. The statement that \verb|mult(m,n )| is correct can be formulated in English as follows:

The program \verb|mult(m,n)| which takes as input any two natural numbers $m,n$ computes the value $mn$ after a finite number of steps.

Write the symbolic statement that is equivalent to the English statement above and prove it (using (1) and (2)).
 
\end{enumerate}


%%Write your solution here
Solution:\vskip5pt
1. According to the precondition, m,n are natural numbers and x = m, which means $x \ge 0$\vskip5pt
Firstly, prove for all natural numbers $k$, $x_k > x_{k+1}\geq 0$\vskip5pt
Case 1: when x = 0, the while loop does not even run. So for any pairs of numbers (n,0), there do exist x = 0 at $I_0$ with x = m = 0\vskip5pt
Case 2: at iteration k, $x_k > 0$. After one more step of while loop, $x_{k+1} = x_k /3$.\vskip5pt
(1) If $x_{k} < 3$, then $x_k//3 = 0 => x_{k+1} = 0$.\vskip5pt
(2) If $x_{k} \ge 3$, then $x_k//3 > 0 => x_{k+1} > 0$\vskip5pt
So that in this case, $x_{k+1} \ge 0$\vskip5pt
Moreover, $x_k/3 < x_k$, which means as k is getting bigger, x is getting smaller. \vskip5pt
Therefore, $x_k > x_{k+1} \ge 0$, holds.
Then it is clear that $x_k > x_{k+1}\geq 0$ for all natural numbers $k$ is true.\vskip5pt
In conclusion, for any for all natural numbers n,m, there exists an iteration k, such that $x_k =0$ at the beginning of $I_k$ \\[2ex]
\\[2ex]
2. We use complete induction to prove \vskip5pt
(1) Base Case: When m = 0, $z_k = 0 - 0 = 0$, holds \vskip5pt
(2) Predicate is P(k): At the end of $I_k$ (line 12), $z_k = mn-x_ky_k$, $\forall k\in \mathbb{N} $\vskip5pt
(3) Induction hypothesis: assume P(k) is true for all natural number k.\vskip5pt
(4) Induction steps: Prove for k + 1:\vskip5pt
if there exits $x_k+1$, then $x_k$ can not be 0\vskip5pt
Case 1: $ 0 < x_k < 3$ so that at iteration $I_{k+1}$, $x_{k+1}$ = $x_k // 3 = 0$\vskip5pt
so that z does not change,which implies to \vskip5pt
$z_{k+1} = mn = mn - 0 = mn - x_{k+1}y_{k+1}$, holds \vskip5pt
Case 2: $x_k = 3,6,9...$, where $x_k$ mod 3 = 0 and we assume $x_k = 3n$,where n is any nature number.\vskip5pt
In this case,$z_{k+1} = z_k = mn - x_ky_k$\vskip5pt
\qquad \qquad \qquad \qquad $= mn - 3ny_k$ (we assumed $x_k = 3n$)\vskip5pt
\qquad \qquad \qquad \qquad $= mn - n(3y_k)$\vskip5pt
\qquad \qquad \qquad \qquad $= mn - x_{k+1}y_{k+1}$ (since $x_{k+1} = x_k // 3 = 3n // 3 = n$ and  $y_{k+1} = 3y_k$)\vskip5pt
Case 3: $ x_k > 3$ and at iteration $I_{k+1}$, $x_k$ mod 3 = 1 and we assume $x_k = 3n + 1$, where n is any nature number .\vskip5pt
So that $z_{k+1} = z_k + y_k$       (by execute the $I_{k+1}$ step in while loop)\vskip5pt
\qquad \qquad \qquad = $mn - x_ky_k + y_k$ (by IH)\vskip5pt
\qquad \qquad \qquad = $mn - (x_k - 1)y_k$\vskip5pt
\qquad \qquad \qquad = $mn - (3n+1-1)y_k$ (since we assumed $x_k = 3n+1$ )\vskip5pt
\qquad \qquad \qquad = $mn - 3ny_k$\vskip5pt
\qquad \qquad \qquad = $mn - n(3y_k)$ \vskip5pt
\qquad \qquad \qquad = $mn - x_{k+1}y_{k+1}$ (since $x_{k+1} = x_{k} // 3 = (3n+1) //3 = n$,line 11 and at $I_{k+1}$, $y_{k+1} = 3*y_k$, line 12), holds\vskip5pt
Case 4: $ x_k > 3$ and at iteration $I_{k+1}$, $x_k$ mod 3 = 2 and we assume $x_k = 3n+2 $, where n is any nature number. \vskip5pt
So that $z_{k+1} = z_k - y_k$  (by execute the $I_{k+1}$ step in while loop )\vskip5pt
\qquad \qquad \qquad = $mn - x_ky_k - y_k$ (by IH)\vskip5pt
\qquad \qquad \qquad = $mn - (x_k +1)y_k$ \vskip5pt
\qquad \qquad \qquad = $mn - (3n+3)y_k$ (since we assumed $x_k = 3n+2$ )\vskip5pt
\qquad \qquad \qquad = $mn - (n+1)(3y_k)$\vskip5pt
\qquad \qquad \qquad = $mn - x_{k+1}y_{k+1}$ (since $x_{k+1} = (x_k+1)//3 = (3n+3)//3 = n+1$ , line 10 and 11. Also $y_{k+1} = 3y_{k}$, line 12), holds\vskip5pt
Therefore, it is true that at the end of $I_k$ (line 12), $z_k = mn-x_ky_k$ \\[2ex]
\\[2ex]
3. According to 2 above, $z_k = mn-x_ky_k$. \vskip5pt
According to 1 above, $x_k$ will decrease as function goes on until $x_k = 0$, which meas the steps are finite. \vskip5pt
So that when $x_k = 0$, the while loop terminates, and we get  $z_k = mn - 0 = mn$, which is the same of postcondition, holds. \vskip5pt
Therefore, the program \verb|mult(m,n)| which takes as input any two natural numbers $m,n$ can compute the value $mn$ after a finite number of steps.


\problem{}
\textsc{(6 marks)} 
 A \emph{palindrome} is a string that is equal to its reversal: examples are 'a', 'wow', and 'abcdedcba'. Consider the following algorithm.
  
  \begin{small}
  \begin{verbatim}
  def longestPalindrome(s):
    ''' 
    Pre: s is a non-empty string
    Post: returns the longest palindrome that is a substring of s.
    If there is more than one palindrome in s of maximum length, 
    return <YOU FIGURE THIS OUT>.
 
    >>> longestPalindrome('ballaaa')
    'alla'
    >>> longestPalindrome('ballaaaa')
    <YOU FIGURE THIS OUT>
    '''
    
    if len(s) == 1:
      return s
    else:
      palindrome1 = longestPalindrome(s[1..len(s)-1])
      palindrome2 = firstPalindrome(s)
      
      if len(palindrome1) > len(palindrome2):
        return palindrome1
      else:
        return palindrome2
  \end{verbatim}
  \end{small}
  
  You have two tasks here, which should be accomplished together.
  \begin{itemize}
    \item As is often the case in real life, the client (Ilir) has failed to consider an edge case in the provided specification. By studying the given algorithm, you must complete the specification.
    \item Once again, write pre- and postconditions for the helper function \texttt{firstPalindrome}, and then prove that \texttt{longestPalindrome(s)} is correct, assuming that \texttt{firstPalindrome} is correct.
  \end{itemize}
  
  Note that you cannot prove that \texttt{longestPalindrome} is correct without completing its specification; but in order to complete its specification, you can \emph{carefully trace through the code}, as you would when actually proving correctness (so the two tasks can be accomplished together).
  
 % \end{enumerate}
  
%%Write your solution here
Solution:\vskip5pt
1. for \texttt{firstPalindrome}\vskip5pt
precondition: the input is a non-empty string\vskip5pt
postcondition: return the first palindrome of the input string from the most left \\[2ex]
\\[2ex]
2. for \texttt{longestPalindrome}\vskip5pt
Postcondition: returns the longest palindrome that is a substring of s. If there is more than one palindrome in s of maximum length, return the rightmost palindrome.\vskip5pt
To prove the correctness of \texttt{longestPalindrome}:\vskip5pt
[1] Path 1: return s\vskip5pt 
When s is a character, the palindrome of s is itself. So it holds for returning s.\vskip5pt
[2] Path 2: return palindrome1 \vskip5pt
\qquad(1) to enter path2 , length of s is greater than 1. So precondition holds\vskip5pt
\qquad(2) s[1..len(s)-1], the input string starts from the second character to the end, the value is getting smaller, so smaller value holds\vskip5pt
\qquad(3) Then we use complete induction to prove:\vskip5pt
\qquad Firstly,  Predicate P(k): \texttt{longestPalindrome} returns longest palindorome of s, where the length of input s is k, $k \ge 1$.\vskip5pt
\qquad Secondly, Induction hypothesis: for $k \ge 1$, \texttt{longestPalindrome} returns longest palindorome of s\vskip5pt
\qquad Thirdly, Induction steps: prove for P(k+1)\vskip5pt
By IH, when s has lenght of k, the longest palindrome is palindrome1. When we add one more string, we compare the palindrome1 with the left most palindrome2 that is generated by \texttt{firstPalindrome}, since in this path, length of palindrome1 is greater than palindrome2, holds\vskip5pt
[3] path 3: return palindrome2\vskip5pt
proof steps are similar to path2, however, to enter path 3, the length of palindrome1 is less than or equal palindrome2. holds\vskip5pt
Finally, check postcondition, if there are no palindromes with same maximum length, we will keep comparing palindrome1 and palindrome2 until the end of input string. But if there are palindromes with same maximum length, we will enter path 3 and get the palindrome whose index is more right than the previous one. holds  

\end{document}